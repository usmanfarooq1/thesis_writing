% 
%            ,,                                        
%          `7MM            _.o9                                
%            MM                                             
%  ,6"Yb.    MM  ,p6"bo   ,6"Yb.  M"""MMV  ,6"Yb.  `7Mb,od8 
% 8)   MM    MM 6M'  OO  8)   MM  '  AMV  8)   MM    MM' "' 
%  ,pm9MM    MM 8M        ,pm9MM    AMV    ,pm9MM    MM     
% 8M   MM    MM YM.    , 8M   MM   AMV  , 8M   MM    MM     
% `Moo9^Yo..JMML.YMbmd'  `Moo9^Yo.AMMmmmM `Moo9^Yo..JMML.   
% 
% 
% Free and Open-Source template for academic works
% https://github.com/dpmj/alcazar

\newpage

\clearpage
\cleardoublepage
\phantomsection

\pagestyle{plain}

\phantomsection
\addcontentsline{toc}{chapter}{Abstract}

{\noindent \large \textbf{\thesisTitle}}\\




{\noindent \textbf{\textsc{Abstract:}}}

\noindent The objective of this thesis is to examine the impact of convolutions on global-scale geospatial data through the utilization of map projections.
To achieve this goal, a comprehensive understanding of the actual shape of the Earth and its relationship to map projections is developed.
In order to analyze the effects, a pipeline is established to convert geo-referenced data into map-projected data in raster format.
Convolutional neural networks (CNN) are used for the study, specifically a U-Net based architecture is used for experimentation.
A thorough investigation is conducted on four categories of map projections to represent the data.
The thesis concludes that, as map projected data is planar (Euclidean space) in nature,
the 2D convolutions would capture the distortions subjected to data by the conversion to map projected data.