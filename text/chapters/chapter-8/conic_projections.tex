\newpage
\section{Conic Projections}
\subsection{Lambert Equal Area Conic}
\begin{figure}[H]
    \centering
    \begin{minipage}{0.30\textwidth}
        \centering
        \includegraphics[width=0.9\linewidth]{figures/chapter-8/geopoth_leac.png}
        \caption{ Geopotential height raster data as Lambert Equal Area Conic projected}
        \label{fig:leac_geopoth_raster}
    \end{minipage}\hfill
    \begin{minipage}{0.30\textwidth}
        \centering
        \includegraphics[width=0.9\linewidth]{figures/chapter-8/leac.png}
        \caption{Lambert Equal Area Conic (Source \cite{PROJ_SITE})}
        \label{fig:leac_proj}
    \end{minipage}\hfill
    \begin{minipage}{0.30\textwidth}
        \centering
        \includegraphics[width=0.9\linewidth]{figures/chapter-8/prect_leac.png}
        \caption{Precipitation raster data as Lambert Equal Area Conic projected}
        \label{fig:leac_prect_raster}
    \end{minipage}\hfill
\end{figure}
\subsection{Albers Equal Area}
\begin{figure}[h]
    \centering
    \begin{minipage}{0.30\textwidth}
        \centering
        \includegraphics[width=0.9\linewidth]{figures/chapter-8/geopoth_aea.png}
        \caption{ Geopotential height raster data as Albers Equal Area projected}
        \label{fig:aea_geopoth_raster}
    \end{minipage}\hfill
    \begin{minipage}{0.30\textwidth}
        \centering
        \includegraphics[width=0.9\linewidth]{figures/chapter-8/aea.png}
        \caption{Albers Equal Area (Source \cite{PROJ_SITE})}
        \label{fig:aea_proj}
    \end{minipage}\hfill
    \begin{minipage}{0.30\textwidth}
        \centering
        \includegraphics[width=0.9\linewidth]{figures/chapter-8/prect_aea.png}
        \caption{Precipitation raster data as Albers Equal Area projected}
        \label{fig:aea_prect_raster}
    \end{minipage}\hfill
\end{figure}
\newpage
\subsection{Vitkovsky I}
\begin{figure}[h]
    \centering
    \begin{minipage}{0.30\textwidth}
        \centering
        \includegraphics[width=0.9\linewidth]{figures/chapter-8/geopoth_vitk.png}
        \caption{ Geopotential height raster data as Vitkovsky I projected}
        \label{fig:vitk_geopoth_raster}
    \end{minipage}\hfill
    \begin{minipage}{0.30\textwidth}
        \centering
        \includegraphics[width=0.9\linewidth]{figures/chapter-8/vitk1.png}
        \caption{Vitkovsky I (Source \cite{PROJ_SITE})}
        \label{fig:vitk_proj}
    \end{minipage}\hfill
    \begin{minipage}{0.30\textwidth}
        \centering
        \includegraphics[width=0.9\linewidth]{figures/chapter-8/prect_vitk.png}
        \caption{Precipitation raster data as Vitkovsky I projected}
        \label{fig:vitk_prect_raster}
    \end{minipage}\hfill
\end{figure}
\subsection{Lambert Conformal Conic Alternative}
\begin{figure}[h]
    \centering
    \begin{minipage}{0.30\textwidth}
        \centering
        \includegraphics[width=0.9\linewidth]{figures/chapter-8/geopoth_lcca.png}
        \caption{ Geopotential height raster data as Lambert Conformal Conic Alternative projected}
        \label{fig:lcca_geopoth_raster}
    \end{minipage}\hfill
    \begin{minipage}{0.30\textwidth}
        \centering
        \includegraphics[width=0.9\linewidth]{figures/chapter-8/lcca.png}
        \caption{Lambert Conformal Conic Alternative (Source \cite{PROJ_SITE})}
        \label{fig:lcca_proj}
    \end{minipage}\hfill
    \begin{minipage}{0.30\textwidth}
        \centering
        \includegraphics[width=0.9\linewidth]{figures/chapter-8/prect_lcca.png}
        \caption{Precipitation raster data as Lambert Conformal Conic Alternative projected}
        \label{fig:lcca_prect_raster}
    \end{minipage}\hfill
\end{figure}

\subsection{Results and Observations}

\begin{table}[ht]
    \centering
    \caption{Summary of Conic Projection Model Performancee}
    \label{conic_results_table}
    \renewcommand{\arraystretch}{1.2} % Adjusts the row height
    \begin{tabular}{|l|c|c|c|c|}
        \hline
        \rowcolor[gray]{0.9}
        \textbf{\emph{Project Name}}        & \textbf{\emph{Epochs}} & \textbf{\emph{MAE}} & \textbf{\emph{Validation MAE}} \\ \hline
        Lambert Equal Area Conic            & 20                     & 0.597               & 0.608                          \\ \hline
        Albers Equal Area                   & 20                     & 0.644               & 0.656                          \\ \hline
        Vitkovsky I                         & 20                     & 0.657               & 0.664                          \\ \hline
        Lambert Conformal Conic Alternative & 20                     & 0.670               & 0.674                          \\ \hline
    \end{tabular}
\end{table}