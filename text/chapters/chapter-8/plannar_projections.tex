
\section{Experiments: Planar Projections}
The selected planar projections for the experimentation are mentioned below:
\begin{itemize}
    \item Azimuthal Equidistant
    \item Wagner VII
    \item Lambert Azimuthal Equal Area
    \item Airy
\end{itemize}

% \subsection{Lambert Azimuthal Equal Area}
\begin{figure}[H]
    \centering
    \begin{minipage}{0.30\textwidth}
        \centering
        \includegraphics[width=0.9\linewidth]{figures/chapter-8/geopoth_laea.png}
        \caption{ Geopotential height raster data as Lambert Azimuthal Equal Area projected}
        \label{fig:laea_geopoth_raster}
    \end{minipage}\hfill
    \begin{minipage}{0.30\textwidth}
        \centering
        \includegraphics[width=0.9\linewidth]{figures/chapter-8/laea.png}
        \caption{Lambert Azimuthal Equal Area (Source \cite{PROJ_SITE})}
        \label{fig:laea_proj}
    \end{minipage}\hfill
    \begin{minipage}{0.30\textwidth}
        \centering
        \includegraphics[width=0.9\linewidth]{figures/chapter-8/prect_laea.png}
        \caption{Precipitation raster data as Lambert Azimuthal Equal Area projected}
        \label{fig:laea_prect_raster}
    \end{minipage}\hfill
\end{figure}

% \subsection{Wagner VII}
% \begin{figure}[H]
%     \centering
%     \begin{minipage}{0.30\textwidth}
%         \centering
%         \includegraphics[width=0.9\linewidth]{figures/chapter-8/geopoth_wag.png}
%         \caption{ Geopotential height raster data as Wagner VII projected}
%         \label{fig:wag_geopoth_raster}
%     \end{minipage}\hfill
%     \begin{minipage}{0.30\textwidth}
%         \centering
%         \includegraphics[width=0.9\linewidth]{figures/chapter-8/wag7.png}
%         \caption{Wagner VII (Source \cite{PROJ_SITE})}
%         \label{fig:wag_proj}
%     \end{minipage}\hfill
%     \begin{minipage}{0.30\textwidth}
%         \centering
%         \includegraphics[width=0.9\linewidth]{figures/chapter-8/prect_wag.png}
%         \caption{Precipitation raster data as Wagner VII projected}
%         \label{fig:wag_prect_raster}
%     \end{minipage}\hfill
% \end{figure}

\subsection{Results and Observations}
\begin{itemize}
    \item The planar projections are generated by a tagent point on the reference surface, making these projection specific to the point at which the projections is generated, even though we want to capture the global data
          it is a challenge using the planar prjections.
    \item The planar projection tries to capture only the specific area of the globe. The rasters generated for the experimentation were subject to create data where it should not be generated.
    \item Due to the above mentioned issues of the planar projections, didn't performed well as compared to the previously mentioned map projections types.
    \item The results are depicted in the table ~\ref{planner_results_table}
\end{itemize}
\begin{table}[ht]
    \caption{Summary of Planar(Azimuthal) Projection Model Performance}
    \label{planner_results_table}
    \renewcommand{\arraystretch}{1.2} % Adjusts the row height
    \begin{tabular}{|l|c|c|c|c|c|}
        \hline
        \rowcolor[gray]{0.9}
        \textbf{\emph{Projection}}   & \textbf{\emph{\# Epochs}} & \textbf{\emph{MAE}} & \textbf{\emph{Validation MAE}} & \textbf{\emph{Loss}} & \textbf{\emph{Validation Loss}} \\ \hline
        Azimuthal Equidistant        & 20                        & 0.621               & 0.634                          & 0.697                & 0.725                           \\ \hline
        Wagner VII                   & 20                        & 0.609               & 0.627                          & 0.677                & 0.710                           \\ \hline
        Lambert Azimuthal Equal Area & 20                        & 0.676               & 0.680                          & 0.798                & 0.803                           \\ \hline
        Airy                         & 20                        & 0.768               & 0.787                          & 0.978                & 0.994                           \\ \hline
    \end{tabular}
\end{table}
