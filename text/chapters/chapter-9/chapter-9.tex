
\clearpage
\cleardoublepage

\chapter{Conclusion \& Future Work}
\label{chap:conclusion_future_work}
In this thesis, we established the understanding regarding the depiction of geospatial data on a global scale, and in order to achieve this, we opted for map projections for the global representation.
We successfully established a comprehensive pipeline for generating map projected geospatial rasters. The map projections are inherently subjected to distortions, bring forward the challenge to accurate data representation and analysis.
The main focus of this study was to see the effects of 2D convolutions on four distinct types of map projections: cylindrical, pseudocylindrical, conic, and planar.

By utilizing a U-Net based architecture that was intentionally designed to be shallow, we were able to concentrate specifically on the convolution effects rather than on the optimization of deep learning performance.

The performance of the cylindrical projections in the experiments surpassed that of the other types of map projections.
The research revealed that although 2D convolutions on map-projected raster data excel in capturing spatial features by taking into account the relationships between neighboring elements,
they inherently convey the distortions specific to their respective projections, thus illustrating the limitations of 2D convolutions.
The primary constraint identified is the insufficiency of two-dimensional convolutions, which operate on Euclidean planes,
to comprehensively tackle the distortions caused by map projections.

\section*{Future Work}
In this era where the geospatial data is being generated constantly and the ever growing need to analysis the geospatial data in various scientific and commercial fields. The need for the true depiction of the geospatial data is direly needed, so the data could be analyzed without the distortions of the map projections.
To analyze geospatial data, now the direction in which has the potential is discussed.
\subsection*{Convolutions On Non-Euclidean Spaces}
The convolutions on the Euclidean plane were not able to mitigate the distortions.
There is a need to move in the direction of the representing the Earth in \textbf{Non Euclidean space}.

\subsubsection*{Sphere}
Spherical CNN \cite{cohen2018spherical} are proposed which are performing convolutions on the sphere. Through this approach,
the representation of the data is much closer to the Earth shape.
Which would be able to mitigate some if not all of the distortions observed by the map projections.


\subsubsection*{Ellipsoid}
While convolutions on sphere are performed using different approaches, these approaches are in some generalized for other purposes as well. For the geospatial data a more precise approach is needed.
Rather than performing convolutions on a sphere, convolutions could be performed on an ellipsoid which could be defined by the reference ellipsoid used for collecting the geospatial data under consideration.
This way the representation of the data would be much more nearer to the Earth shape than a sphere, making the usage specific to geospatial data.

