
\clearpage
\cleardoublepage

\chapter{Conclusion and Future Work}
\label{chap:conclusion_future_work}
In this thesis, we established the understanding regarding the depiction of geospatial data on a global scale, and in order to achieve this, we opted for map projections for the global representation.
We successfully established a comprehensive pipeline for generating map projected geospatial rasters. The map projections are inherently subjected to distortions, bring forward the challenge to accurate data representation and analysis.
The main focus of this study to see the effects of 2D convolutions on four distinct types of map projections: cylindrical, pseudocylindrical, conic, and planar.

By utilizing a U-Net based architecture that was intentionally designed to be shallow, we were able to concentrate specifically on the convolution effects rather than on the optimization of deep learning performance.

The performance of the cylindrical projections in the experiments surpassed that of the other types of map projections.
The research revealed that although 2D convolutions on map-projected raster data excel in capturing spatial features by taking into account the relationships between neighboring elements,
they inherently convey the distortions specific to their respective projections, thus illustrating the limitations of 2D convolutions. The primary constraint identified is the insufficiency of two-dimensional convolutions, which operate on Euclidean planes, to comprehensively tackle the distortions caused by map projections.